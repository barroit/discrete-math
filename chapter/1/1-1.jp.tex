% SPDX-License-Identifier: MPL-2.0

\subsection{集合}

集合は属するか否かが判定できる対象の集まり

\begin{topic}{表し方}
   \begin{tabular}{ll}
      列挙     & $A = \set{ 1, 2, 3, 4, 5 }$ \\
      省略記法 & $B = \set{ 5, 10, 15, \,\ldots }$ \\
      条件付き & $C = \set[n = 5k,\, k \, は自然数]{ n }$
   \end{tabular}
\end{topic}

\begin{topic}{記号}
   \begin{tabular}{ll}
      集合 & $A, B, C$ \\
      要素 & $a, b, c$
   \end{tabular}

   \begin{tabular}{ll}
      対象全体 & $U$ \\
      空集合   & $\varnothing$
   \end{tabular}

   \begin{tabular}{ll}
      $a$は$A$の要素       & $a \in A$ \\
      $a$は$A$の要素でない & $a \notin A$
   \end{tabular}

   \begin{tabular}{ll}
      $A$は$B$の部分集合       & $A \subseteq B$ \\
      $A$は$B$の部分集合でない & $A \not\subseteq B$
   \end{tabular}
\end{topic}

\begin{formulatopic}{部分集合}
   A \subseteq B \iff \forall x \,(x \in A \Rightarrow x \in B)
\end{formulatopic}

\begin{itemtopic}[部分集合の性質]
   \item $U \subseteq U$
   \item $\forall A \,(\varnothing \subseteq A)$
   \item $A \subseteq B \land B \subseteq A \Rightarrow A = B$
\end{itemtopic}

\begin{formulatopic}{真部分集合}
   A \subset B \iff A \subseteq B \land A \neq B
\end{formulatopic}

\begin{itemtopic}[真部分集合の代表的な包含関係の例]
   \item $\mathbb{N} \subset \mathbb{Z} \subset
          \mathbb{Q} \subset \mathbb{R} \subset \mathbb{C}$
\end{itemtopic}

\begin{formulatopic}{ベキ集合}
   \mathcal{P}(A) = \set{X \mid X \subseteq A}
\end{formulatopic}

\begin{itemtopic}
   \item 別記号 $2^A$
\end{itemtopic}

\begin{topic}{主な集合}
   \begin{tabular}{lll}
      $\mathbb{N}$ & 自然数 & natural numbers \\
      $\mathbb{Z}$ & 整数   & integers \\
      $\mathbb{Q}$ & 有理数 & rational numbers \\
      $\mathbb{R}$ & 実数   & real numbers \\
      $\mathbb{C}$ & 複素数 & complex numbers
   \end{tabular}
\end{topic}
